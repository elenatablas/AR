%%%%%%%%%%%% Encaminamiento intra-dominio IPv4 - ORGANIZACIÓN B  %%%%%%%%%%%%
\section{Organización B}
\par En este apartado explicaré cómo he configurado los routers de la organización B para que utilicen el protocolo de encaminamiento \textbf{OSPF (Open Shortest Path First)} para la construcción de sus tablas de rutas.
\par Este protocolo se adapta mejor a los cambios que RIP a la vez que soporta tanto \texttt{subnetting} como \texttt{CIDR}.
\par Para habilitar OSPF en cada uno de los routers, he tenido que introducir los siguientes comandos:
%%% CÓDIGO %%%
\begin{listing}[style=consola]
Router> enable
Router# configure terminal
Router(config)# router ospf 100
Router(config-router)# network MASCARA area AREA_ID
\end{listing}
\par Introduciendo un comando network con los parámetros adecuados para cada subred a las que estuviera directamente conectado el router. Por ejemplo, en el caso del \texttt{RouterB1}:
\begin{listing}[style=consola]
network 173.89.9.132 0.0.0.3 area 0
network 173.89.9.136 0.0.0.3 area 0
network 173.89.10.0 0.0.0.127 area 1
\end{listing}
\par Al igual que en la organización A para RIP, aquí también he introducido el siguiente comando en los routers conectados a subredes de área local (RouterB(0,2,3,4 y 5)) por las que no quiero que se envíen mensajes LSA del protocolo OSPF:
\begin{listing}[style=consola]
Router(config-router)# passive-interface FastEthernet0/0
\end{listing}
\par Por último, también quiero destacar la configuración adicional que he introducido en el RouterB1 y RouterB2, los ABRs en la red de la organización B para conseguir reducir el número de entradas en las tablas de rutas gracias a la \textbf{agregación de rutas}.
\par Para ello, he utilizado los siguientes comandos:
\begin{listing}[style=consola]
RouterB1(config-router)# area 0 range 173.89.8.0 255.255.254.0
RouterB1(config-router)# area 1 range 173.89.10.0 255.255.255.0
RouterB2(config-router)# area 0 range 173.89.8.0 255.255.254.0
RouterB2(config-router)# area 2 range 173.89.11.0 255.255.255.0
\end{listing}
\par Después de introducirlos observo cómo se reduce el número de entradas en las tablas de rutas de forma considerable, puesto que los routers de la organización ya no necesitan conocer cómo llegar a cada subred de otra área si no que es suficiente saber cómo llegar a esa área.
\par Este último cambio es posible gracias a que al realizar el direccionamiento de la organización B, he hecho un esfuerzo por repartir el espacio de direcciones de forma que, después, las subredes dentro de la misma área se pudieran resumir sin que se produjeran colisiones.

%%%%%%%%%%%% Encaminamiento intra-dominio IPv4 - organización B %%%%%%%%%%%%
\begin{ejer}
1.[OrgB] Realice la configuración necesaria para que RouterB3 se convierta
en Designated Router (DR) de la LAN 2.2.
\end{ejer}
%%%%%%%%%%%% Encaminamiento intra-dominio IPv4 - organización B  %%%%%%%%%%%%
\begin{ejer}
2.[OrgB] Muestre las tablas de rutas de RouterB3 y comente los aspectos más
relevantes. ¿Cuál es el camino óptimo para alcanzar RouterB4?
\end{ejer}
%%%%%%%%%%%% INTERCONEXIÓN %%%%%%%%%%%%
\begin{ejer}
3. Tras la redistribución consulte las tablas de rutas de los routers del Área 1 para demostrar que se trata de una totally stub área. ¿Qué sucede con la tabla de rutas? ¿Por qué?
\end{ejer}
%%%%%%%%%%%% INTERCONEXIÓN %%%%%%%%%%%%
\begin{ejer}
4. Consulte también las tablas de rutas de los routers del Área 2 y explique por qué se trata de un área stub. ¿Qué ocurriría en el caso de que no fuera stub? ¿Por qué?
\end{ejer}
\par Para este ejercicio considero las tablas de rutas de los routers RouterB4 y RouterB5:
\begin{listing}[style=consola]
RouterB4>sh ip route
     173.89.0.0/16 is variably subnetted, 6 subnets, 5 masks
O IA    173.89.8.0/23 [110/65] via 173.89.11.193, 00:27:26, Serial2/0
O IA    173.89.10.0/24 [110/130] via 173.89.11.193, 00:27:36, Serial2/0
O       173.89.11.0/25 [110/65] via 173.89.11.198, 00:28:32, Serial3/0
C       173.89.11.128/26 is directly connected, FastEthernet0/0
C       173.89.11.192/30 is directly connected, Serial2/0
C       173.89.11.196/30 is directly connected, Serial3/0
O*IA 0.0.0.0/0 [110/65] via 173.89.11.193, 00:28:32, Serial2/0
\end{listing}

\begin{listing}[style=consola]
RouterB5>sh ip route
     173.89.0.0/16 is variably subnetted, 6 subnets, 5 masks
O IA    173.89.8.0/23 [110/129] via 173.89.11.197, 00:27:50, Serial2/0
O IA    173.89.10.0/24 [110/194] via 173.89.11.197, 00:27:50, Serial2/0
C       173.89.11.0/25 is directly connected, FastEthernet0/0
O       173.89.11.128/26 [110/65] via 173.89.11.197, 00:28:46, Serial2/0
O       173.89.11.192/30 [110/128] via 173.89.11.197, 00:28:46, Serial2/0
C       173.89.11.196/30 is directly connected, Serial2/0
O*IA 0.0.0.0/0 [110/129] via 173.89.11.197, 00:28:36, Serial2/0
\end{listing}
\par A diferencia de en la pregunta anterior aquí se puede ver como sí aparecen rutas hacia redes de otras áreas, pero sigue sin haber rutas hacia redes externas (no hay ninguna entrada que comience por O E2).
\par Ambos routers tienen una ruta por defecto al ABR, el router RouterB2. Esto y lo anterior nos lleva a pensar que el área 2 es un \texttt{área stub}.
%%%%%%%%%%%% Encaminamiento intra-dominio IPv4 - organización B  %%%%%%%%%%%%
\begin{ejer}
5.[OrgB]  Realice la configuración necesaria para que el camino óptimo entre RouterB3 y RouterB4 pase a través de RouterB0.
\end{ejer}
\par Para conseguir esto es necesario aumentar el coste de la interfaz Serial 3/0 del router RouterB1 para que pase a tener un coste mayor que la suma de los costes de pasar por el router RouterB0 que sería 64 + 64 = 128.
Entonces será suficiente con configurar la interfaz para que tenga un coste de 129, lo que podemos hacer con la siguiente secuencia de comandos:
\begin{listing}[style=consola]
RouterB1>enable
RouterB1#configure terminal
RouterB1(config)#interface Serial3/0
RouterB1(config-if)#ip ospf cost 129
\end{listing}
\par Compruebo que la modificación se ha llevado a cabo correctamente: \\
\begin{listing}[style=consola]
RouterB1#sh ip ospf interface Serial3/0
Serial3/0 is up, line protocol is up
  Internet address is 173.89.9.137/30, Area 0
  Process ID 100, Router ID 173.89.10.1, Network Type POINT-TO-POINT, Cost: 129
  Transmit Delay is 1 sec, State POINT-TO-POINT,
  Timer intervals configured, Hello 10, Dead 40, Wait 40, Retransmit 5
    Hello due in 00:00:01
  Index 2/2, flood queue length 0
  Next 0x0(0)/0x0(0)
  Last flood scan length is 1, maximum is 1
  Last flood scan time is 0 msec, maximum is 0 msec
  Neighbor Count is 1 , Adjacent neighbor count is 1
    Adjacent with neighbor 173.89.11.193
  Suppress hello for 0 neighbor(s)
\end{listing}
Y mediante un traceroute ver si sigue el camino esperado:
\begin{listing}[style=consola]
RouterB3#traceroute 173.89.11.194
Tracing the route to 173.89.11.194

  1   173.89.10.1     0 msec    0 msec    0 msec    
  2   173.89.9.133    12 msec   0 msec    0 msec    
  3   173.89.9.138    15 msec   3 msec    31 msec   
  4   173.89.11.194   55 msec   72 msec   1 msec
\end{listing}
%%%%%%%%%%%% Encaminamiento intra-dominio IPv4 - organización B  %%%%%%%%%%%%
\begin{ejer}
6.[OrgB]  Realice la configuración necesaria para que el área 2 sea una stub area. Analizando las tablas de rutas que considere relevantes, ¿qué diferencias observa con respecto a la configuración anterior? ¿Por qué?
\end{ejer}
\par Para que el área 2 sea una \texttt{stub area} es necesario realizar la siguiente configuración en todos los routers del área (RouterB2, RouterB4 y RouterB5):
\begin{listing}[style=consola]
RouterB2>enable
RouterB2#configure terminal
RouterB2(config)#router ospf 100
RouterB2(config-router)#area 2 stubr
\end{listing}
\par La tabla de rutas del router RouterB2 no ha sufrido ningún cambio. Mientras tanto las tablas de rutas de los routers RouterB4 y RouterB5 sí. El cambio en la tabla de rutas de los routers involucrados ha sido el mismo en ambas, por resumir voy a coger como ejemplo únicamente la salida de la tabla de rutas del router RouterB4:
\begin{listing}[style=consola]
RouterB4#show ip route
     173.89.0.0/16 is variably subnetted, 6 subnets, 4 masks
O IA    173.89.8.0/24 [110/129] via 173.89.11.193, 00:02:34, Serial2/0
O IA    173.89.10.0/25 [110/129] via 173.89.11.193, 00:02:34, Serial2/0
O       173.89.11.0/25 [110/65] via 173.89.11.198, 00:02:06, Serial3/0
C       173.89.11.128/26 is directly connected, FastEthernet0/0
C       173.89.11.192/30 is directly connected, Serial2/0
C       173.89.11.196/30 is directly connected, Serial3/0
O*IA 0.0.0.0/0 [110/65] via 173.89.11.193, 00:02:34, Serial2/0
\end{listing}
\par Como se puede ver el único cambio está en la última línea donde vemos que se ha añadido un candidato por defecto que coincide con el ABR, el router RouterB2.\\
\par En la tabla de rutas del router R12 la línea que se ha añadido es la siguiente: 
\begin{listing}[style=consola]
O*IA 0.0.0.0/0 [110/129] via 173.89.11.197, 00:02:13, Serial2/0
\end{listing}
\par Ahora todos los paquetes cuya dirección de destino pertenezca a una red externa se enviarán hacia el ABR ya que en un \texttt{área stub} se filtran los LSAs de tipo 4 y 5 y, por tanto, los routers del área no conocen los caminos directos a las direcciones de redes externas.


%%%%%%%%%%%% Encaminamiento intra-dominio IPv4 - organización B  %%%%%%%%%%%%
\begin{ejer}
7.[OrgB]  Deshabilite la interfaz del router RouterB0 que conecta con el RouterB2. Espere a que la red converja de nuevo. A continuación, realiza el traceroute de nuevo entre RouterB0 y HostB1 y justifica el camino que ahora siguen los paquetes.
\end{ejer}
%%%%%%%%%%%% Encaminamiento intra-dominio IPv4 - organización B  %%%%%%%%%%%%
\newpage
\begin{ejer}
8.[OrgB] Utilizando la herramienta Packet Tracer capture tráfico OSPF para mostrar al menos dos tipos de LSA diferentes que se intercambian los routers del escenario e indique su propósito.
\end{ejer}
\par Puedo capturar el tráfico OSPF que pasa por un router con el siguiente comando:
\begin{listing}[style=consola]
Router# debug ip ospf events
\end{listing}
\par Para el ejercicio voy a coger ejemplos a partir del router RouterB1:

\begin{listing}[style=consola]
00:00:10: OSPF: Build router LSA for area 1, router ID 173.89.10.1, seq 0xffffffff80000002
\end{listing}

\par Este sería un ejemplo de \textbf{LSA de tipo 1}. El router RouterB1 (router ID: 173.89.10.1) construye un mensaje para anunciar su presencia en, en este caso, el área 1. Realmente se tendría que enviar un LSA de tipo 1 por cada una de las áreas a las que está conectado, que en este caso son dos. Este trabajo lo tendrían que hacer todos los routers.

\begin{listing}[style=consola]
00:00:40: OSPF: Build summary LSA for area 0, router ID 173.89.10.1, seq 0x80000008
\end{listing}

\par Éste sería un ejemplo de \textbf{LSA de tipo 3}. El router RouterB1 (router ID: 173.89.10.1) al ser el ABR (Area Border Router) es el encargado de generar un LSA de este tipo por cada subred de un área y las envía a los routers de otras áreas.