%%%%%%%%%%%% INTERCONEXIÓN %%%%%%%%%%%%
\begin{ejer}
1. Muestre las tablas de rutas de los routers RouterA2 y RouterB0 y coméntelas en detalle.
\end{ejer}

\begin{listing}[style=consola]
RouterA2>show ip route
     173.89.0.0/16 is variably subnetted, 21 subnets, 6 masks
R       173.89.0.0/23 [120/2] via 173.89.3.197, 00:00:11, Serial0/0/0
R       173.89.2.0/24 [120/3] via 173.89.3.197, 00:00:11, Serial0/0/0
R       173.89.3.0/25 [120/2] via 173.89.3.197, 00:00:11, Serial0/0/0
C       173.89.3.128/26 is directly connected, GigabitEthernet0/0
L       173.89.3.130/32 is directly connected, GigabitEthernet0/0
R       173.89.3.192/30 [120/1] via 173.89.3.197, 00:00:11, Serial0/0/0
C       173.89.3.196/30 is directly connected, Serial0/0/0
L       173.89.3.198/32 is directly connected, Serial0/0/0
R       173.89.3.200/30 [120/2] via 173.89.3.197, 00:00:11, Serial0/0/0
R       173.89.3.204/30 [120/1] via 173.89.3.197, 00:00:11, Serial0/0/0
R       173.89.3.212/30 [120/2] via 173.89.3.197, 00:00:11, Serial0/0/0
C       173.89.3.216/30 is directly connected, Serial0/1/0
L       173.89.3.217/32 is directly connected, Serial0/1/0
R       173.89.8.0/24 [120/1] via 173.89.3.218, 00:00:11, Serial0/1/0
R       173.89.9.0/25 [120/1] via 173.89.3.218, 00:00:11, Serial0/1/0
R       173.89.9.132/30 [120/1] via 173.89.3.218, 00:00:11, Serial0/1/0
R       173.89.9.136/30 [120/1] via 173.89.3.218, 00:00:11, Serial0/1/0
R       173.89.10.0/24 [120/1] via 173.89.3.218, 00:00:07, Serial0/1/0
R       173.89.10.0/25 is possibly down, routing via 173.89.3.218, Serial0/1/0
R       173.89.11.0/24 is possibly down, routing via 173.89.3.218, Serial0/1/0
R       173.89.11.0/25 [120/1] via 173.89.3.218, 00:00:07, Serial0/1/0
\end{listing}
\par En la tabla de rutas del router RouterA2 puedo ver que, a diferencia de antes de hacer la redistribución de rutas, ahora hay entradas para llegar a cualquier subred tanto de la organización A como de la organización B.
\par Para las rutas que se han redistribuido desde OSPF no aparece el número de routers real que hay que atravesar para llegar a una subred. Mientras que para las rutas que se han obtenido directamente por el protocolo RIP, es decir, las rutas a aquellas subredes dentro de la organización A, sí.
\par También puedo destacar que en la tabla de rutas del router RouterA2 siguen manteniendo las agregaciones de rutas que hice de las áreas 1 y 2. Es decir, en lugar de aparecer una entrada por cada subred de dichas áreas, únicamente aparece una entrada por cada área con la dirección agregada.
\begin{listing}[style=consola]
RouterB0>show ip route

     173.89.0.0/16 is variably subnetted, 16 subnets, 5 masks
R       173.89.0.0/23 [120/3] via 173.89.3.217, 00:00:20, Serial2/0
R       173.89.2.0/24 [120/4] via 173.89.3.217, 00:00:20, Serial2/0
R       173.89.3.0/25 [120/3] via 173.89.3.217, 00:00:20, Serial2/0
R       173.89.3.128/26 [120/1] via 173.89.3.217, 00:00:20, Serial2/0
R       173.89.3.192/30 [120/2] via 173.89.3.217, 00:00:20, Serial2/0
R       173.89.3.196/30 [120/1] via 173.89.3.217, 00:00:20, Serial2/0
R       173.89.3.200/30 [120/3] via 173.89.3.217, 00:00:20, Serial2/0
R       173.89.3.204/30 [120/2] via 173.89.3.217, 00:00:20, Serial2/0
R       173.89.3.212/30 [120/3] via 173.89.3.217, 00:00:20, Serial2/0
C       173.89.3.216/30 is directly connected, Serial2/0
C       173.89.8.0/24 is directly connected, FastEthernet0/0
O       173.89.9.0/25 [110/194] via 173.89.9.134, 00:06:10, Serial4/0
C       173.89.9.132/30 is directly connected, Serial4/0
O       173.89.9.136/30 [110/193] via 173.89.9.134, 00:06:20, Serial4/0
O IA    173.89.10.0/24 [110/66] via 173.89.9.134, 00:05:44, Serial4/0
O IA    173.89.11.0/24 [110/257] via 173.89.9.134, 00:05:29, Serial4/0
\end{listing}
\par En la tabla de rutas del router RouterB0 puedo ver cómo conviven los dos protocolos OSPF y RIP en el mismo router.
\par Las entradas que comienzan con R corresponden con las rutas obtenidas mediante el protocolo RIP, es decir, todas las de la organización A. Mientras que las que comienzan con O son las subredes de la organización A que están en la misma área que el router RouterB0, el área 0. Las entradas que comienzan por O IA son respectivamente las direcciones de las áreas 1 y 2 de OSPF.



