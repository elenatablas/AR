%Import package

%Magenes
\usepackage[margin=1 in, includefoot]{geometry}

% Acentos, letras, etc
\usepackage[T1]{fontenc}
\usepackage[utf8]{inputenc}
\usepackage[spanish]{babel}
\addto\captionsspanish{
\renewcommand{\chaptername}{}
}
\selectlanguage{spanish}
\usepackage{lettrine, Zallman}
\renewcommand\LettrineFontHook{\Zallmanfamily}

% Caja de texto
\usepackage{parskip}

\usepackage{times}
\usepackage{ulem} % texto tachado
\usepackage{lipsum}
\usepackage[usenames]{color} % color en texto
%Creación de mis propios colores
\definecolor{rojoOscuro}{RGB}{153,0,0}
\definecolor{verdeOscuro}{RGB}{0,153,0}
\usepackage{verbatim} % comentarios
% Letra para código
\usepackage{listings}
\lstdefinestyle{consola}
{basicstyle=\scriptsize\bf\ttfamily,
backgroundcolor=\color{gray75},
}

% Enumerados
%\renewcommand{\theenumi}{\Alph{enumi}} %Letras mayúsculas
%\renewcommand{\labelenumi}{{\theenumi})}
\renewcommand{\labelenumii}{\alph{enumii}$)$ }

\usepackage{comment}

% Header and Footer Stuff
\usepackage{fancyhdr}

%hipervínculos, referencias, citas, etc
\usepackage[colorlinks=true, 
		     linkcolor=magenta, 
		     citecolor = black,
		     urlcolor = blue]{hyperref}
\pagestyle{fancy}
\fancyhead[L,RO]{}
\fancyhead[LO,R]{}
%fancy
\renewcommand{\footrulewidth}{1pt}

% Espacios en los títulos
%\usepackage{titlesec}

% Tablas de contenidos separadas, Ramon Sánchez y Pedro M. Ruiz
% https://www.ctan.org/pkg/minitoc
%\usepackage[spanish]{minitoc}

% Símbolos matemáticos
\usepackage{amsmath}
\usepackage{array}

% Figuras, gráficas, tablas y matrices
\usepackage{float}
\usepackage{subfigure} % varias figuras
\usepackage{graphicx} 
\usepackage{longtable}
%\graphicspath{ {images/} }


% Color en las tablas
\usepackage{color}
\usepackage{epsfig}
\usepackage{multirow}
\usepackage{colortbl}
\usepackage[table]{xcolor}
\usepackage{soul}

%https://ondahostil.wordpress.com/2017/05/17/lo-que-he-aprendido-cuadros-de-texto-de-colores-en-latex/
\usepackage{lmodern}
\usepackage{tcolorbox}
\tcbuselibrary{listingsutf8}

% Definir cuadro de ancho del texto
\newtcolorbox{mybox}[1]{colback=purple!5!white,colframe=purple!75!black,fonttitle=\bfseries,title=#1}
% OPCIONES
% colback: color de fondo
% colframe: color de borde
% fonttitle: estilo de título
% title: título de la cuadro o referencia a argumento

\usepackage{incgraph}

% Cuadro estrecho
\newtcbox{cuadro}[1]{colback=blue!5!white,colframe=blue!75!black,fonttitle=\bfseries,title=#1}

% Cuadro numerado para ejemplos
\newtcolorbox[auto counter,number within=subsection]{example}[2][]
{colback=purple!5!white,colframe=purple!75!black,fonttitle=\bfseries, title=Ejemplo~\thetcbcounter: #2,#1}

\newtcolorbox[auto counter,number within=subsection]{ejercicio}[2][]
{colback=orange!5!white,colframe=orange!75!black,fonttitle=\bfseries, title=#2,#1}

\newtcolorbox[auto counter,number within=subsection]{solucion}[2][]
{colback=yellow!5!white,colframe=yellow!75!black,fonttitle=\bfseries, title=#2,#1}

% Esquemas con llaves https://tex.stackexchange.com/questions/164664/how-to-create-an-array-with-both-vertical-and-horizontal-braces-around-the-eleme
\newcommand\undermat[2]{%
\makebox[0pt][l]{$\smash{\underbrace{\phantom{%
\begin{matrix}#2\end{matrix}}}_{\text{$#1$}}}$}#2}

% CheckBox-todo-list
%https://tex.stackexchange.com/questions/247681/how-to-create-checkbox-todo-list
\usepackage{enumitem,amssymb}
\newlist{todolist}{itemize}{2}
\setlist[todolist]{label=$\square$}
\usepackage{pifont}
\newcommand{\cmark}{\ding{51}}%
\newcommand{\xmark}{\ding{55}}%
\newcommand{\done}{\rlap{$\square$}{\raisebox{2pt}{\large\hspace{1pt}\cmark}}%
\hspace{-2.5pt}}
\newcommand{\wontfix}{\rlap{$\square$}{\large\hspace{1pt}\xmark}}

% Licencia
\usepackage[
    type={CC},
    modifier={by-nc-sa},
    version={3.0},
]{doclicense}

% Letra para código http://www.rafalinux.com/?p=599
\definecolor{gray97}{gray}{.97}
\definecolor{gray75}{gray}{.75}
\definecolor{gray45}{gray}{.45}
\definecolor{pblue}{rgb}{0.13,0.13,1}
\definecolor{pgreen}{rgb}{0,0.5,0}
\definecolor{pred}{rgb}{0.9,0,0}
\definecolor{pgrey}{rgb}{0.46,0.45,0.48}

\usepackage{listings}
\lstset{ 
	frame=Ltb,
	framerule=0pt,
	aboveskip=0.5cm,
	framextopmargin=3pt,
	framexbottommargin=3pt,
	framexleftmargin=0.2cm,
	framesep=0pt,
	rulesep=.4pt,
	backgroundcolor=\color{gray97},
	rulesepcolor=\color{black},
%
	stringstyle=\ttfamily,
	showstringspaces = false,
	basicstyle=\small\ttfamily,
	commentstyle=\color{gray45},
	keywordstyle=\bfseries,
%
	numbers=left,
	numbersep=7pt,
	numberstyle=\tiny,
	numberfirstline = false,
	breaklines=true,
}

% minimizar fragmentado de listados
\lstnewenvironment{listing}[1][]
{\lstset{#1}\pagebreak[0]}{\pagebreak[0]}

\lstdefinestyle{consola}
{basicstyle=\scriptsize\bf\ttfamily,
backgroundcolor=\color{gray75},
}

\usepackage{inconsolata}
% https://tex.stackexchange.com/questions/115467/listings-highlight-java-annotations

\lstset{language=Java,
  showspaces=false,
  showtabs=false,
  breaklines=true,
  showstringspaces=false,
  breakatwhitespace=true,
  commentstyle=\color{pgreen},
  keywordstyle=\color{pblue},
  stringstyle=\color{pred},
  basicstyle=\ttfamily,
  moredelim=[il][\textcolor{pgrey}]{$$},
  moredelim=[is][\textcolor{pgrey}]{\%\%}{\%\%}
}

%Cuestionario
\usepackage{multicol}
\definecolor{verdeOscuro}{rgb}{0.05, 0.5, 0.06}

\newcommand{\Width}{\textwidth}
\newcommand{\HalfWidth}{0.46\textwidth}
\newcommand{\Height}{0.95\textheight}
\newcommand{\HalfHeight}{0.45\textheight}
\newcommand{\QuarterHeight}{0.20\textheight}
\newcommand{\HorizontalLine}
{
  \noindent\makebox[\linewidth]{\rule{\Width}{0.4pt}}
}
\newcommand{\IndentOn}{\setlength{\parindent}{12pt}}
\newcommand{\IndentOff}{\setlength{\parindent}{0pt}}

\newcounter{QuestionCounter}
\setcounter{QuestionCounter}{1}

\newenvironment{QuestionnaireQuestions}
{
  \begin{small}
    \setlength{\columnsep}{1cm}
    \setlength{\parskip}{12pt plus 0pt minus 9pt} % ELASTIC SPACE BETWEEN QUESTIONS
    \noindent \begin{multicols}{2}
}
{
  \end{multicols}
  \end{small}
  \newpage
}
\newcommand\QuestionnaireComment[1] 
{
  \fbox{
		\begin{minipage}{\HalfWidth}
			\small{\underline{Razonamiento}: \textcolor{red}{ #1 }}
		\end{minipage}} 
}
\newenvironment{QuestionnaireQuestion}[1][]
{
  \begin{minipage}{\HalfWidth}
    \HorizontalLine\\
    \tiny{\textbf{\arabic{QuestionCounter}. #1}}
    \begin{scriptsize}
    \begin{enumerate}[label=\alph*)]
}
{
  \\
  \end{enumerate}
  \end{scriptsize}
  \end{minipage}
  \stepcounter{QuestionCounter}
}
%Llaves
\usepackage{schemata}
% Foto
\usepackage{wrapfig}
% Tablas
\usepackage{multirow}
\usepackage{tikz}
\def\checkmark{\tikz\fill[scale=0.4](0,.35) -- (.25,0) -- (1,.7) -- (.25,.15) -- cycle;} 

%https://ondahostil.wordpress.com/2017/05/17/lo-que-he-aprendido-cuadros-de-texto-de-colores-en-latex/
\usepackage{lmodern}
\usepackage{tcolorbox}
\tcbuselibrary{listingsutf8}

%%%%%% %%%%%% PREAMBULO %%%%%% %%%%%%

\setlength{\parindent}{0.5cm}
%\setlength{\parskip}{\baselineskip} 
\lhead[\leftmark]{Pérez González-Tablas, Elena}
\rhead[Nombre Autor]{Práctica 4. Scrum, 2021/22}
\graphicspath{ {images/} }
% Cuadro estrecho
\newtcbox{cua}[1]{colback=blue!5!white,colframe=blue!75!black,fonttitle=\bfseries,title=#1}

% Cuadro numerado para ejemplos
\newtcolorbox[auto counter]{ejer}[2][]
{colback=yellow!5!white,colframe=yellow!90!black,fonttitle=\bfseries\color{black}, title=~\thetcbcounter: #2,#1}

\newtcolorbox[auto counter]{anverso}{colback=yellow!40!white,colframe=black, arc=4mm,sharp corners=northwest,sharp corners=south, arc is angular}

\newtcolorbox{reverso}{colback=yellow!40!white,colframe=black, arc=4mm,sharp corners=northeast,sharp corners=south, arc is angular}
