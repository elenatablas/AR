%%%%%%%%%%%% INTRODUCCIÓN  %%%%%%%%%%%%

\begin{center}
	{\fboxrule=4pt \fbox{\fboxrule=1pt
		\fbox{\LARGE{\bfseries 4. Conclusiones y valoración personal}}}} \\
	\addcontentsline{toc}{chapter}{4. Conclusiones y valoración personal}
	\setcounter{chapter}{4}
	\setcounter{section}{0}
	\rule{15cm}{0pt} \\
\end{center}
 
\par Este trabajo es muy completo, he podido conocer aspectos básicos de la estructura de una red a nivel hardware insertando distintos tipos de routers, hosts y cables. Incluso más específicamente, añadiendo los módulos en los routers que no venían por defecto.
\par Me ha servido para afianzar los conocimientos teóricos de esta asignatura y la asignatura de redes del año pasado. Configurar la red es un trabajo costoso y puede acarrear muchas erratas difíciles de ver. En cambio, realizar las cuestiones me ha servido para plantearme algunas dudas y demostrar que realmente domino y entiendo cómo funciona la red que he creado, no es solo insertar unos comandos y probar que funciona.
\par El manejo de Cisco Packet Tracer es muy cómodo porque es multiplataforma, aunque me hubiera gustado que se pudieran compartir las topologías en tiempo real.
\par Los vídeos, tutorías electrónicas, clases y boletines me han sido de gran utilidad porque esta práctica está muy bien guiada y esquematizada. Agradezco en particular a mi profesor de prácticas del año pasado por la ayuda proporcionada.
\par Considero que no he encontrado ninguna dificultad adicional gracias a todo el material disponible ya mencionado.
\par La realización de esta memoria es el grueso del tiempo empleado en estas prácticas, uno de los motivos es el estricto formato. En total he realizado 30 horas de trabajo autónomo.