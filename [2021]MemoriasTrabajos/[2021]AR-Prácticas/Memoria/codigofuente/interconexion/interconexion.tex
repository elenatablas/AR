%%%%%%%%%%%% INTRODUCCIÓN  %%%%%%%%%%%%

\begin{center}
	{\fboxrule=4pt \fbox{\fboxrule=1pt
		\fbox{\LARGE{\bfseries 3. Interconexión y redistribución de rutas}}}} \\
	\addcontentsline{toc}{chapter}{3. Interconexión y redistribución de rutas}
	\setcounter{chapter}{3}
	\setcounter{section}{0}
	\rule{15cm}{0pt} \\
\end{center}
\par En este apartado voy a redistribuir las rutas porque estoy interconectando dos organizaciones que tienen distintos protocolos de enrutamiento y es necesario intercambiar información de enrutamiento para que los distintos elementos de la red se puedan comunicar.
\par En esta topología voy a configurar el RouterB0 que interconecta las dos organizaciones.
\begin{itemize}
	\item Redistribución de rutas de OSPF a RIP:
%%% CÓDIGO %%%
\begin{listing}[style=consola]
RouterB0(config)# router rip
RouterB0(config-router)# redistribute ospf 100 metric 1
RouterB0(config-router)# passive-interface FastEthernet0/0
RouterB0(config-router)# passive-interface Serial3/0
RouterB0(config-router)# passive-interface Serial4/0
RouterB0(config-router)# exit
\end{listing}
	\item Redistribución de rutas de RIP a OSPF:
%%% CÓDIGO %%%
\begin{listing}[style=consola]
RouterB0(config)# router ospf 100
RouterB0(config-router)# redistribute rip metric 200 subnets 
RouterB0(config-router)# passive-interface Serial2/0
RouterB0(config-router)# exit
\end{listing}
\end{itemize}
\par El comando \texttt{network} no sería necesario puesto que ya fue configurado en apartados anteriores. El comando \texttt{passive-interface} evita que al redistribuir las rutas se vuelva a enviar por las interfaces donde ya está funcionando el protocolo, información redundante.

%%%%%%%%%%%% Encaminamiento intra-dominio IPv4 - organización B %%%%%%%%%%%%
\begin{ejer}
1.[OrgB] Realice la configuración necesaria para que RouterB3 se convierta
en Designated Router (DR) de la LAN 2.2.
\end{ejer}
%%%%%%%%%%%% Encaminamiento intra-dominio IPv4 - organización B  %%%%%%%%%%%%
\begin{ejer}
2.[OrgB] Muestre las tablas de rutas de RouterB3 y comente los aspectos más
relevantes. ¿Cuál es el camino óptimo para alcanzar RouterB4?
\end{ejer}
%%%%%%%%%%%% INTERCONEXIÓN %%%%%%%%%%%%
\begin{ejer}
3. Tras la redistribución consulte las tablas de rutas de los routers del Área 1 para demostrar que se trata de una totally stub área. ¿Qué sucede con la tabla de rutas? ¿Por qué?
\end{ejer}
%%%%%%%%%%%% INTERCONEXIÓN %%%%%%%%%%%%
\begin{ejer}
4. Consulte también las tablas de rutas de los routers del Área 2 y explique por qué se trata de un área stub. ¿Qué ocurriría en el caso de que no fuera stub? ¿Por qué?
\end{ejer}
\par Para este ejercicio considero las tablas de rutas de los routers RouterB4 y RouterB5:
\begin{listing}[style=consola]
RouterB4>sh ip route
     173.89.0.0/16 is variably subnetted, 6 subnets, 5 masks
O IA    173.89.8.0/23 [110/65] via 173.89.11.193, 00:27:26, Serial2/0
O IA    173.89.10.0/24 [110/130] via 173.89.11.193, 00:27:36, Serial2/0
O       173.89.11.0/25 [110/65] via 173.89.11.198, 00:28:32, Serial3/0
C       173.89.11.128/26 is directly connected, FastEthernet0/0
C       173.89.11.192/30 is directly connected, Serial2/0
C       173.89.11.196/30 is directly connected, Serial3/0
O*IA 0.0.0.0/0 [110/65] via 173.89.11.193, 00:28:32, Serial2/0
\end{listing}

\begin{listing}[style=consola]
RouterB5>sh ip route
     173.89.0.0/16 is variably subnetted, 6 subnets, 5 masks
O IA    173.89.8.0/23 [110/129] via 173.89.11.197, 00:27:50, Serial2/0
O IA    173.89.10.0/24 [110/194] via 173.89.11.197, 00:27:50, Serial2/0
C       173.89.11.0/25 is directly connected, FastEthernet0/0
O       173.89.11.128/26 [110/65] via 173.89.11.197, 00:28:46, Serial2/0
O       173.89.11.192/30 [110/128] via 173.89.11.197, 00:28:46, Serial2/0
C       173.89.11.196/30 is directly connected, Serial2/0
O*IA 0.0.0.0/0 [110/129] via 173.89.11.197, 00:28:36, Serial2/0
\end{listing}
\par A diferencia de en la pregunta anterior aquí se puede ver como sí aparecen rutas hacia redes de otras áreas, pero sigue sin haber rutas hacia redes externas (no hay ninguna entrada que comience por O E2).
\par Ambos routers tienen una ruta por defecto al ABR, el router RouterB2. Esto y lo anterior nos lleva a pensar que el área 2 es un \texttt{área stub}.
\newpage