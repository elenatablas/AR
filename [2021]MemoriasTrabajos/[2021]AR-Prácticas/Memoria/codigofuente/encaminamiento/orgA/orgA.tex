%%%%%%%%%%%% Encaminamiento intra-dominio IPv4 - organización A  %%%%%%%%%%%%

\section{Organización A}

\par Voy a configurar y administrar las subredes de la organización A con el protocolo de encaminamiento RIP.
\par En cada router hay que poner estos comandos:
%%% CÓDIGO %%%
\begin{listing}[style=consola]
Router> enable
Router# configure terminal
Router(config)# router rip
Router(config-router)# version 2
Router(config-router)# network 173.89.0.0
Router(config-router)# no auto-summary
\end{listing}
\par Utilizo la \textbf{versión 2} de RIP porque estoy implementando \textbf{subnetting} y \textbf{VLSM (variable length subnet masks)} que permite que las máscaras sean de tamaño variable. También, me aseguro de que \textbf{no realice agregación automática de rutas}, porque quiero que en cada vector distancia aparezca una entrada por cada subred en vez de sólo la dirección de red agregada.
\par Como todos los routers conectan subredes que pertenecen a la misma red IP que es la 173.89.0.0/22, no es necesario indicar cada subred conectada directamente a los routers. Indicando simplemente la \textbf{dirección de red classful}, la implementación se encargará de anunciar cada subred de forma independiente en cada router.
\par Además, en los routers que están conectados a las subredes \texttt{LAN} incluyo este comando, porque son redes que sólo contienen hosts y dichos hosts no ejecutan RIP, puesto que he configurado manualmente sus rutas. Las interfaces de los routers que están conectadas a estas subredes actuarán de \textbf{forma ``pasiva''} respecto a \texttt{RIP} para no obtener un gasto inútil de ancho de banda.
\par En el router \texttt{RouterA0} 
\begin{listing}[style=consola]
RouterA0(config-router)# passive-interface GigabitEthernet0/0
RouterA0(config-router)# passive-interface GigabitEthernet0/1
\end{listing}
\par En los routers \texttt{RouterA2, RouterA3 y RouterA4}
\begin{listing}[style=consola]
Router(config-router)# passive-interface GigabitEthernet0/0
\end{listing}

%%%%%%%%%%%% Encaminamiento intra-dominio IPv4 - organización B %%%%%%%%%%%%
\begin{ejer}
1.[OrgB] Realice la configuración necesaria para que RouterB3 se convierta
en Designated Router (DR) de la LAN 2.2.
\end{ejer}
%%%%%%%%%%%% Encaminamiento intra-dominio IPv4 - organización B  %%%%%%%%%%%%
\begin{ejer}
2.[OrgB] Muestre las tablas de rutas de RouterB3 y comente los aspectos más
relevantes. ¿Cuál es el camino óptimo para alcanzar RouterB4?
\end{ejer}
%%%%%%%%%%%% INTERCONEXIÓN %%%%%%%%%%%%
\begin{ejer}
3. Tras la redistribución consulte las tablas de rutas de los routers del Área 1 para demostrar que se trata de una totally stub área. ¿Qué sucede con la tabla de rutas? ¿Por qué?
\end{ejer}


\newpage