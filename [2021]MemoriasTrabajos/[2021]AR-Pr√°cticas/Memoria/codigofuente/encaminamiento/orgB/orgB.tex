%%%%%%%%%%%% Encaminamiento intra-dominio IPv4 - ORGANIZACIÓN B  %%%%%%%%%%%%
\section{Organización B}
\par En este apartado explicaré cómo he configurado los routers de la organización B para que utilicen el protocolo de encaminamiento \textbf{OSPF (Open Shortest Path First)} para la construcción de sus tablas de rutas.
\par Este protocolo se adapta mejor a los cambios que RIP a la vez que soporta tanto \texttt{subnetting} como \texttt{CIDR}.
\par Para habilitar OSPF en cada uno de los routers, he tenido que introducir los siguientes comandos:
%%% CÓDIGO %%%
\begin{listing}[style=consola]
Router> enable
Router(config)# router ospf 100
sRouter(config-router)# network MASCARA area AREA_ID
\end{listing}
\par Introduciendo un comando network con los parámetros adecuados para cada subred a las que estuviera directamente conectado el router. Por ejemplo, en el caso del \texttt{RouterB1}: //TODO
\begin{listing}[style=consola]
network 173.89.10.132 0.0.0.3 area 0
network 173.89.10.136 0.0.0.3 area 0
network 173.89.12.0 0.0.0.127 area 1
network 173.89.13.200 0.0.0.3 area 2
network 173.89.13.192 0.0.0.3 area 2
\end{listing}
\par Al igual que en la organización A para RIP, aquí también he introducido el siguiente comando en los routers conectados a subredes de área local (RouterB(1,2,3,4)//TODO) por las que no queremos que se envíen mensajes LSA del protocolo OSPF:
\begin{listing}[style=consola]
Router(config-router)# passive-interface FastEthernet 0/0
\end{listing}
\par Por último, también quiero destacar la configuración adicional que hemos introducido en el RouterB//TODO, el ABR en la red de la organización B para conseguir reducir el número de entradas en las tablas de rutas gracias a la \textbf{agregación de rutas}.
\par Para ello, he utilizado los siguientes comandos:
\begin{listing}[style=consola]
RouterB(config-router)# area 0 range 173.89.8.0 255.255.252.0
RouterB(config-router)# area 2 range 172.59.12.0 255.255.255.0
RouterB(config-router)# area 2 range 173.89.13.0 255.255.255.0
\end{listing}
\par Después de introducirlos observamos cómo se reduce el número de entradas en las tablas de rutas de forma considerable, puesto que los routers de la organización ya no necesitan conocer cómo llegar a cada subred de otra área si no que es suficiente saber cómo llegar a esa área.
\par Este último cambio es posible gracias a que al realizar el direccionamiento de la organización B, he hecho un esfuerzo por repartir el espacio de direcciones de forma que, después, las subredes dentro de la misma área se pudieran resumir sin que se produjeran colisiones.

%%%%%%%%%%%% Encaminamiento intra-dominio IPv4 - organización A  %%%%%%%%%%%%
\begin{ejer}
1.[OrgA] Muestre las tablas de rutas de RouterA3 y comente los aspectos más relevantes. ¿Cuál es el camino óptimo para alcanzar la interfaz de RouterA2 que conecta con la Organización B? ¿Por qué? ¿Cuántas alternativas hay para alcanzarlo según la tabla de rutas?
\end{ejer}

%%%%%%%%%%%% INTERCONEXIÓN %%%%%%%%%%%%
\newpage
\begin{ejer}
2. Realice un traceroute del host HostA2 al HostB4. Explica y justifica el camino que se sigue. Indica cómo es posible que el RouterA1 que utiliza un protocolo de enrutamiento intra-dominio, puede obtener información de otro SA distinto que utiliza otro protocolo de enrutamiento intra-dominio distinto.
\end{ejer}

\begin{listing}[style=consola]
C:\>tracert 173.89.11.130

Tracing route to 173.89.11.130 over a maximum of 30 hops: 

  1   21 ms     0 ms      0 ms      173.89.2.1
  2   0 ms      0 ms      49 ms     173.89.3.213
  3   96 ms     95 ms     0 ms      173.89.3.205
  4   1 ms      2 ms      1 ms      173.89.3.198
  5   57 ms     2 ms      100 ms    173.89.3.218
  6   4 ms      46 ms     1 ms      173.89.9.134
  7   91 ms     112 ms    208 ms    173.89.9.138
  8   126 ms    5 ms      2 ms      173.89.11.194
  9   58 ms     89 ms     3 ms      173.89.11.130

Trace complete.
\end{listing}
\par El camino que sigue es el siguiente: HostA2 $\rightarrow$ RouterA4 $\rightarrow$ RouterA3 $\rightarrow$ RouterA1 $\rightarrow$ RouterA2 $\rightarrow$ RouterB0 $\rightarrow$ RouterB1 $\rightarrow$ RouterB2 $\rightarrow$ RouterB4 $\rightarrow$ HostB4.
\par Sigue este camino porque en el ejercicio 3 de la organización A del apartado anterior (RIP) se pide que se desactive en RouterA3 la interfaz de salida hacia RouterA2. Como ahora ya no es posible ir por ese camino, va por un camino alternativo que pasa por RouterA1.
\par Al llegar al RouterB0 en lugar de ir directamente al RouterB2, pasa por el router RouterB1 porque en el ejercicio 5 de la organización B del apartado anterior (OSPF) se pidió que el camino óptimo entre RouterB0 y RouterB2 pasase por RouterB1 y por tanto el camino óptimo entre RouterB0 y RouterB2 aquí también pasa por RouterB1: \textbf{Principio de optimalidad}.
\par RouterA4 utiliza el protocolo de intradominio RIP, sin embargo, gracias a la redistribución de rutas configurada en el router Router0, he conseguido que la información de encaminamiento a redes de la organización B calculadas mediante OSPF se transmita mediante RIP a los routers de la organización A.
%%%%%%%%%%%% Encaminamiento intra-dominio IPv4 - organización A  %%%%%%%%%%%%
\begin{ejer}
3.[OrgA] Empleando el comando tracert, muestre la ruta que sigue el tráfico desde el HostA2 hasta la interfaz de RouterA2 que conecta con la Organización B. ¿Qué pasa si lo hacemos a la interfaz del RouterB0 en la red P2P1.6?
\par Con la simulación en marcha, desactive en RouterA3 la interfaz de salida hacia RouterA2. Utilizando información de las tablas de rutas y capturas del tráfico RIP en la red (Packet Tracer y/o salida de debug de los routers Cisco), explique en detalle cómo RIP converge a una nueva solución para alcanzar RouterA2. Céntrese únicamente en los routers RouterA3 y RouterA2.
\par Indique, en caso de que aplique, el funcionamiento sobre este escenario y el uso de las técnicas triggered updates y poison reverse. \\
\end{ejer}

\par Consola del HostA2, donde muestro la ruta que sigue el tráfico hasta la interfaz de RouterA3 que conecta con la Organización B: 

\begin{listing}[style=consola]
C:\>tracert 173.89.3.217

Tracing route to 173.89.3.217 over a maximum of 30 hops: 

  1   0 ms      10 ms     0 ms      173.89.2.1
  2   14 ms     27 ms     0 ms      173.89.3.213
  3   18 ms     0 ms      14 ms     173.89.3.217

Trace complete.
\end{listing}
\par Consola del HostA2, donde muestro la ruta que sigue el tráfico hasta la interfaz de RouterB0 que conecta con la Organización B, puedo comprobar que hay un salto más porque traspasa RouterA3:
\begin{listing}[style=consola]
C:\>tracert 173.89.3.218

Tracing route to 173.89.3.218 over a maximum of 30 hops: 

  1   0 ms      0 ms      0 ms      173.89.2.1
  2   0 ms      0 ms      19 ms     173.89.3.213
  3   16 ms     2 ms      34 ms     173.89.3.209
  4   26 ms     45 ms     48 ms     173.89.3.218

Trace complete.
\end{listing} 
\par Primero miro las \textbf{características} del protocolo \texttt{RIP} que se está usando en este router. Realiza un envío periódico de mensajes \texttt{RIP Response} cada 30 segundos de su vector de distancias actualizado. Si durante 180 segundos no se ha vuelto a saber de una determinada red destino, ésta se elimina de la \texttt{routing database}. Puedo observar que no implementa la característica \textbf{triggered updates} porque la tabla de las interfaces que están conectadas a este router que implementan el protocolo \texttt{RIP} no hay ninguna información en la columna \texttt{Triggered RIP}.
\begin{listing}[style=consola]
Routing Protocol is "rip"
Sending updates every 30 seconds, next due in 24 seconds
Invalid after 180 seconds, hold down 180, flushed after 240
Outgoing update filter list for all interfaces is not set
Incoming update filter list for all interfaces is not set
Redistributing: rip
Default version control: send version 2, receive 2
  Interface             Send  Recv  Triggered RIP  Key-chain
  Serial0/1/1           2     2     
  Serial0/0/1           2     2     
  Serial0/1/0           2     2     
  Serial0/0/0           2     2
\end{listing}
\par Una vez sé cómo funciona el protocolo en cada interfaz de este router voy a ver algunos mensajes que se envían y se reciben cuando desactivo la interfaz \texttt{Serial0/1/0}.
\begin{listing}[style=consola]
RouterA3# debug ip rip
RIP protocol debugging is on
RouterA3#configure terminal
RouterA3(config)# interface Serial0/1/0
RouterA3(config-if)# shutdown
RIP: sending  v2 update to 224.0.0.9 via Serial0/1/1 (173.89.3.213)
RIP: build update entries
      173.89.0.0/23 via 0.0.0.0, metric 2, tag 0
      173.89.3.0/25 via 0.0.0.0, metric 1, tag 0
      173.89.3.128/26 via 0.0.0.0, metric 2, tag 0
      173.89.3.192/30 via 0.0.0.0, metric 2, tag 0
      173.89.3.196/30 via 0.0.0.0, metric 2, tag 0
      173.89.3.200/30 via 0.0.0.0, metric 1, tag 0
      173.89.3.204/30 via 0.0.0.0, metric 1, tag 0
      173.89.3.216/30 via 0.0.0.0, metric 16, tag 0
      173.89.8.0/24 via 0.0.0.0, metric 16, tag 0
      173.89.9.0/25 via 0.0.0.0, metric 16, tag 0
      173.89.9.128/30 via 0.0.0.0, metric 16, tag 0
      173.89.9.132/30 via 0.0.0.0, metric 16, tag 0
      173.89.9.136/30 via 0.0.0.0, metric 16, tag 0
      173.89.10.0/24 via 0.0.0.0, metric 16, tag 0
      173.89.11.0/24 via 0.0.0.0, metric 16, tag 0
RIP: sending  v2 update to 224.0.0.9 via Serial0/0/1 (173.89.3.206)
RIP: build update entries
      173.89.0.0/23 via 0.0.0.0, metric 2, tag 0
      173.89.2.0/24 via 0.0.0.0, metric 2, tag 0
      173.89.3.0/25 via 0.0.0.0, metric 1, tag 0
      173.89.3.128/26 via 0.0.0.0, metric 2, tag 0
      173.89.3.200/30 via 0.0.0.0, metric 1, tag 0
      173.89.3.212/30 via 0.0.0.0, metric 1, tag 0
      173.89.3.216/30 via 0.0.0.0, metric 16, tag 0
      173.89.8.0/24 via 0.0.0.0, metric 16, tag 0
      173.89.9.0/25 via 0.0.0.0, metric 16, tag 0
      173.89.9.128/30 via 0.0.0.0, metric 16, tag 0
      173.89.9.132/30 via 0.0.0.0, metric 16, tag 0
      173.89.9.136/30 via 0.0.0.0, metric 16, tag 0
      173.89.10.0/24 via 0.0.0.0, metric 16, tag 0
      173.89.11.0/24 via 0.0.0.0, metric 16, tag 0
RIP: sending  v2 update to 224.0.0.9 via Serial0/0/0 (173.89.3.202)
RIP: build update entries
      173.89.2.0/24 via 0.0.0.0, metric 2, tag 0
      173.89.3.0/25 via 0.0.0.0, metric 1, tag 0
      173.89.3.196/30 via 0.0.0.0, metric 2, tag 0
      173.89.3.204/30 via 0.0.0.0, metric 1, tag 0
      173.89.3.212/30 via 0.0.0.0, metric 1, tag 0
      173.89.3.216/30 via 0.0.0.0, metric 16, tag 0
      173.89.8.0/24 via 0.0.0.0, metric 16, tag 0
      173.89.9.0/25 via 0.0.0.0, metric 16, tag 0
      173.89.9.128/30 via 0.0.0.0, metric 16, tag 0
      173.89.9.132/30 via 0.0.0.0, metric 16, tag 0
      173.89.9.136/30 via 0.0.0.0, metric 16, tag 0
      173.89.10.0/24 via 0.0.0.0, metric 16, tag 0
      173.89.11.0/24 via 0.0.0.0, metric 16, tag 0
\end{listing}
\par Cuando desactivo la interfaz, observo que ocho redes se quedan inalcanzables porque el único camino que tenía desde este router hasta estas ocho subredes, 173.89.3.216/30, 173.89.8.0/24, 173.89.9.0/25, 173.89.9.128/30, 173.89.9.132/30, 173.89.9.136/30, 173.89.10.0/24, 173.89.11.0/24, era mediante esa interfaz como he comentado en el primer ejercicio.
\par Cuando monitorizo los mensajes \texttt{RIP} veo que sigue esperando los 30 segundos como si fuera otro envío periódico, por tanto aseguro que no usa la característica \texttt{triggered updates} porque no recibe \texttt{request, ack}, ni empieza el tiempo de retransmisión. En cambio, si está activado \texttt{poisson reverse}, las redes inalcanzables se anuncian con coste infinito, que en el caso de RIP es 16, como se puede ver en las entradas de esas dos IPs.
\begin{listing}[style=consola]
RouterA3# no debug ip rip
RouterA3#show ip route
	173.89.0.0/16 is variably subnetted, 21 subnets, 6 masks
R       173.89.0.0/23 [120/1] via 173.89.3.201, 00:00:08, Serial0/0/0
R       173.89.2.0/24 [120/1] via 173.89.3.214, 00:00:24, Serial0/1/1
C       173.89.3.0/25 is directly connected, GigabitEthernet0/0
L       173.89.3.1/32 is directly connected, GigabitEthernet0/0
R       173.89.3.128/26 [120/1] via 173.89.3.201, 00:00:08, Serial0/0/0
R       173.89.3.192/30 [120/1] via 173.89.3.201, 00:00:08, Serial0/0/0
                        [120/1] via 173.89.3.205, 00:00:11, Serial0/0/1
R       173.89.3.196/30 [120/1] via 173.89.3.205, 00:00:11, Serial0/0/1
C       173.89.3.200/30 is directly connected, Serial0/0/0
L       173.89.3.202/32 is directly connected, Serial0/0/0
C       173.89.3.204/30 is directly connected, Serial0/0/1
L       173.89.3.206/32 is directly connected, Serial0/0/1
C       173.89.3.212/30 is directly connected, Serial0/1/1
L       173.89.3.213/32 is directly connected, Serial0/1/1
R       173.89.3.216/30 [120/2] via 173.89.3.205, 00:00:11, Serial0/0/1
R       173.89.8.0/24 [120/3] via 173.89.3.205, 00:00:11, Serial0/0/1
R       173.89.9.0/25 [120/3] via 173.89.3.205, 00:00:11, Serial0/0/1
R       173.89.9.128/30 [120/3] via 173.89.3.205, 00:00:11, Serial0/0/1
R       173.89.9.132/30 [120/3] via 173.89.3.205, 00:00:11, Serial0/0/1
R       173.89.9.136/30 [120/3] via 173.89.3.205, 00:00:11, Serial0/0/1
R       173.89.10.0/24 [120/3] via 173.89.3.205, 00:00:11, Serial0/0/1
R       173.89.11.0/24 [120/3] via 173.89.3.205, 00:00:11, Serial0/0/1
\end{listing}
\par En la tabla de rutas se han eliminado las entradas de las IPs que estaban conectadas por esa interfaz y se ha calculado las rutas óptimas con los \texttt{mensajes RIP} recibidos de las otras interfaces de red, como se podía ver en la monitorización de antes. Como resultado, la ruta óptima para alcanzar el RouterA2 depende de la interfaz a la que queremos acceder, como se muestra en la tabla de RouterA3 y con los siguientes comandos:
\begin{listing}[style=consola]
RouterA3# traceroute 173.89.3.130
Tracing the route to 173.89.3.130

  1   173.89.3.201    10 msec   14 msec   15 msec   
  2   173.89.3.130    26 msec   48 msec   23 msec
  
RouterA3# traceroute 173.89.3.198
Tracing the route to 173.89.3.198

  1   173.89.3.205    10 msec   19 msec   23 msec   
  2   173.89.3.198    39 msec   21 msec   26 msec
  
RouterA3# traceroute 173.89.3.217
Tracing the route to 173.89.3.217

  1   173.89.3.205    8 msec    1 msec    18 msec   
  2   173.89.3.198    38 msec   34 msec   36 msec

RouterA3# traceroute 173.89.3.209
Type escape sequence to abort.
Tracing the route to 173.89.3.209

  1   *     *     *     
  2   *     *     *     
  3   *     *     *     
  4   *     *     *   
  ...  
  30  *     *     *   
\end{listing}








%%%%%%%%%%%% Encaminamiento intra-dominio IPv4 - organización B  %%%%%%%%%%%%
\begin{ejer}
4.[OrgB] Muestre las tablas de rutas de RouterB4 y comente los aspectos más relevantes.
\end{ejer}
%%%%%%%%%%%% Encaminamiento intra-dominio IPv4 - organización B  %%%%%%%%%%%%
\begin{ejer}
5.[OrgB]  Realice la configuración necesaria para que el camino óptimo entre RouterB3 y RouterB4 pase a través de RouterB0.
\end{ejer}
%%%%%%%%%%%% Encaminamiento intra-dominio IPv4 - organización B  %%%%%%%%%%%%
\begin{ejer}
6.[OrgB]  Realice la configuración necesaria para que el área 2 sea una stub area. Analizando las tablas de rutas que considere relevantes, ¿qué diferencias observa con respecto a la configuración anterior? ¿Por qué?
\end{ejer}
\par Para que el área 2 sea una \texttt{stub area} es necesario realizar la siguiente configuración en todos los routers del área (RouterB2, RouterB4 y RouterB5):
\begin{listing}[style=consola]
RouterB2>enable
RouterB2#configure terminal
RouterB2(config)#router ospf 100
RouterB2(config-router)#area 2 stubr
\end{listing}
\par La tabla de rutas del router RouterB2 no ha sufrido ningún cambio. Mientras tanto las tablas de rutas de los routers RouterB4 y RouterB5 sí. El cambio en la tabla de rutas de los routers involucrados ha sido el mismo en ambas, por resumir voy a coger como ejemplo únicamente la salida de la tabla de rutas del router RouterB4:
\begin{listing}[style=consola]
RouterB4#show ip route
     173.89.0.0/16 is variably subnetted, 6 subnets, 4 masks
O IA    173.89.8.0/24 [110/129] via 173.89.11.193, 00:02:34, Serial2/0
O IA    173.89.10.0/25 [110/129] via 173.89.11.193, 00:02:34, Serial2/0
O       173.89.11.0/25 [110/65] via 173.89.11.198, 00:02:06, Serial3/0
C       173.89.11.128/26 is directly connected, FastEthernet0/0
C       173.89.11.192/30 is directly connected, Serial2/0
C       173.89.11.196/30 is directly connected, Serial3/0
O*IA 0.0.0.0/0 [110/65] via 173.89.11.193, 00:02:34, Serial2/0
\end{listing}
\par Como se puede ver el único cambio está en la última línea donde vemos que se ha añadido un candidato por defecto que coincide con el ABR, el router RouterB2.\\
\par En la tabla de rutas del router R12 la línea que se ha añadido es la siguiente: 
\begin{listing}[style=consola]
O*IA 0.0.0.0/0 [110/129] via 173.89.11.197, 00:02:13, Serial2/0
\end{listing}
\par Ahora todos los paquetes cuya dirección de destino pertenezca a una red externa se enviarán hacia el ABR ya que en un \texttt{área stub} se filtran los LSAs de tipo 4 y 5 y, por tanto, los routers del área no conocen los caminos directos a las direcciones de redes externas.


%%%%%%%%%%%% Encaminamiento intra-dominio IPv4 - organización B  %%%%%%%%%%%%
\begin{ejer}
7.[OrgB]  Deshabilite la interfaz del router RouterB0 que conecta con el RouterB2. Espere a que la red converja de nuevo. A continuación, realiza el traceroute de nuevo entre RouterB0 y HostB1 y justifica el camino que ahora siguen los paquetes.
\end{ejer}
\par Para deshabilitar la interfaz del router RouterB0 que conecta con RouterB2 hay que introducir los siguientes comandos:
\begin{listing}[style=consola]
RouterB0> enable
RouterB0# configure terminal
RouterB0(config)# interface Serial3/0
RouterB0(config-if)# shutdown
\end{listing}
\par El traceroute entre RouterB0 y HostB1 produce la siguiente salida:
\begin{listing}[style=consola]
RouterB0> traceroute 173.89.9.2
Tracing the route to 173.89.9.2

  1   173.89.9.134    0 msec    13 msec   11 msec   
  2   173.89.9.138    24 msec   23 msec   10 msec   
  3   173.89.9.2      20 msec   1 msec    14 msec  
\end{listing}
\par En lugar de pasar directamente de RouterB0 a RouterB2 y de ahí al host, ahora tiene que cambiar su recorrido a: RouterB0 $\rightarrow$ \textbf{RouterB1} $\rightarrow$ RouterB2 $\rightarrow$ HostB1, ya que la interfaz está desactivada.



%%%%%%%%%%%% Encaminamiento intra-dominio IPv4 - organización B  %%%%%%%%%%%%
\begin{ejer}
8.[OrgB] Utilizando la herramienta Packet Tracer capture tráfico OSPF para mostrar al menos dos tipos de LSA diferentes que se intercambian los routers del escenario e indique su propósito.
\end{ejer}