%%%%%%%%%%%% Encaminamiento intra-dominio IPv4 - organización B  %%%%%%%%%%%%
\begin{ejer}
5.[OrgB]  Realice la configuración necesaria para que el camino óptimo entre RouterB3 y RouterB4 pase a través de RouterB0.
\end{ejer}
\par Para conseguir esto es necesario aumentar el coste de la interfaz Serial 3/0 del router RouterB1 para que pase a tener un coste mayor que la suma de los costes de pasar por el router RouterB0 que sería 64 + 64 = 128.
Entonces será suficiente con configurar la interfaz para que tenga un coste de 129, lo que podemos hacer con la siguiente secuencia de comandos:
\begin{listing}[style=consola]
RouterB1>enable
RouterB1#configure terminal
RouterB1(config)#interface Serial3/0
RouterB1(config-if)#ip ospf cost 129
\end{listing}
\par Compruebo que la modificación se ha llevado a cabo correctamente: \\
\begin{listing}[style=consola]
RouterB1#sh ip ospf interface Serial3/0
Serial3/0 is up, line protocol is up
  Internet address is 173.89.9.137/30, Area 0
  Process ID 100, Router ID 173.89.10.1, Network Type POINT-TO-POINT, Cost: 129
  Transmit Delay is 1 sec, State POINT-TO-POINT,
  Timer intervals configured, Hello 10, Dead 40, Wait 40, Retransmit 5
    Hello due in 00:00:01
  Index 2/2, flood queue length 0
  Next 0x0(0)/0x0(0)
  Last flood scan length is 1, maximum is 1
  Last flood scan time is 0 msec, maximum is 0 msec
  Neighbor Count is 1 , Adjacent neighbor count is 1
    Adjacent with neighbor 173.89.11.193
  Suppress hello for 0 neighbor(s)
\end{listing}
Y mediante un traceroute ver si sigue el camino esperado:
\begin{listing}[style=consola]
RouterB3#traceroute 173.89.11.194
Tracing the route to 173.89.11.194

  1   173.89.10.1     0 msec    0 msec    0 msec    
  2   173.89.9.133    12 msec   0 msec    0 msec    
  3   173.89.9.138    15 msec   3 msec    31 msec   
  4   173.89.11.194   55 msec   72 msec   1 msec
\end{listing}