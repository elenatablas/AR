%%%%%%%%%%%% Encaminamiento intra-dominio IPv4 - organización B  %%%%%%%%%%%%
\begin{ejer}
3.[OrgB] Realice la configuración necesaria para que el área 1 sea una totally stub area. Analizando las tablas de rutas que considere relevantes, demuestre que se trata de una totally stub área. ¿Qué diferencias observa con respecto a la configuración anterior? ¿Por qué?
\end{ejer}
\par El \texttt{área 1} se puede transformar en una \texttt{totally stub area} porque la única alternativa para llegar a las otras áreas es pasar por el router RouterB1 que es el \texttt{ABR} de este área.
\par He introducido los siguientes comandos en la consola del router RouterB3:

\begin{listing}[style=consola]
RouterB3>enable
RouterB3#configure terminal
RouterB3(config)#router ospf 100
RouterB3(config-router)#area 1 stub
\end{listing}

\par Añado los siguientes comandos en la consola del router RouterB1 que tiene un parámetro adicional el último comando porque es el ABR de este área:
\begin{listing}[style=consola]
RouterB1>enable
RouterB1#configure terminal
RouterB1(config)#router ospf 100
RouterB1(config-router)#area 1 stub no-summarys
\end{listing}
\par El ABR no deja pasar hacia dentro del área los LSAs de tipo 4, tipo 5 ni los de tipo 3. En su lugar el ABR genera un LSA de tipo 3 anunciando una ruta por defecto. Por tanto, en la tabla de rutas del router interno, R10, sólo hay entradas para poder llegar a las redes dentro del propio área, C, y una ruta por defecto, *, para cualquier otra red tanto de otras áreas, IA, como externas.
\begin{listing}[style=consola]
RouterB3#show ip route 
     173.89.0.0/16 is variably subnetted, 2 subnets, 2 masks
C       173.89.10.0/25 is directly connected, FastEthernet1/0
C       173.89.10.128/27 is directly connected, FastEthernet0/0
O*IA 0.0.0.0/0 [110/2] via 173.89.10.1, 00:00:22, FastEthernet1/0
\end{listing}