%%%%%%%%%%%% DIRECIONAMIENTO ORGANIZACIÓN B  %%%%%%%%%%%%
	
\section{Configuración en Cisco Packet Tracer.}

\par El primer paso para configurar estas dos organizaciones en \texttt{Cisco Packet Tracer} es establecer la configuración de los hosts y la de las interfaces de los routers. En los hosts indico la IP del \texttt{Gateway} por defecto que será la dirección IP de la interfaz del router a la que está conectado, la IP del host y su máscara de red. Estos pasos los hago manualmente mediante la interfaz gráfica en la sección config. En cambio, para configurar los routers utilizaré sus terminales.
\par Ejemplo de configuración inicial de un router:
%%% CÓDIGO %%%
\begin{listing}[style=consola]
Router> enable
Router# configure terminal
Router(config)# hostname Router
Router(config)# interface IFNAME
Router(config-if)# no shutdown
Router(config-if)# ip address IP_ADDR MASK 
Router(config-if)# exit
Router(config)# exit
Router# write
\end{listing}
\par Primero entro en modo privilegiado, después, en modo configurador. Cambio el nombre del router para mayor legibilidad y configuro cada interfaz insertando su IP, máscara de red correspondiente y activando cada interfaz. Una vez configurado el router tengo que guardar los cambios en memoria \texttt{NVRAM}.