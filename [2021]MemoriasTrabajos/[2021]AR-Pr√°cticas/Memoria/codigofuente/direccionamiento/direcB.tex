%%%%%%%%%%%% DIRECIONAMIENTO ORGANIZACIÓN B  %%%%%%%%%%%%
	
\section{Organización B.}

\par La red de la organización B tiene asignado el rango de direcciones IP 173.89.8.0/22. En este caso sigo el mismo procedimiento que en el apartado anterior.
\par La única diferencia estará en la forma que se lleva a cabo el reparto de direcciones: en lugar de empezar asignando direcciones a las subredes más grandes, voy a realizar el reparto tratando de minimizar el número de entradas en las tablas de rutas mediante la agregación.
\par Es decir, reparto las direcciones de forma que las subredes dentro de una misma área se puedan ``resumir'' o agregar bajo una misma dirección de red con un máscara mayor, teniendo en cuenta que las direcciones no se solapen entre sí.
\par Como consecuencia aparecerán huecos en el espacio de direcciones que sin embargo serán útiles en el caso de que en el futuro quisiéramos incluir nuevas subredes dentro de cada área.
\par Esto está relacionado con el concepto de \textbf{supernetting} estudiado en la teoría en el que se permite que para varias redes contiguas a las que se accede por el mismo salto se puedan resumir en una sola entrada en la tabla de rutas, en lugar de tener que incluir una para cada subred.
\definecolor{amarillo}{rgb}{0.99, 0.93, 0.0}
%%%% TABLA DE SUBREDES DE LA ORGANIZACIÓN B %%%%
\begin{table}[H]
    \centering
    \begin{tabular}{|c|c|c|c|}
        \hline
       	Subred&Número de Interfaces&Bits SubNetID&Bits HostID\\\hline
	\cellcolor{rojo}{LAN2.0}&225&2&8 \\ \hline
	\cellcolor{naranja}{LAN2.1}&125&3&7 \\ \hline
	\cellcolor{lila}{LAN2.2}&115&3&7 \\ \hline
	\cellcolor{rosa}{LAN2.3}&15&5&5 \\ \hline
	\cellcolor{amarillo}{LAN2.4}&120&3&7 \\ \hline
	\cellcolor{azul}{LAN2.5}&60&4&6 \\ \hline
	\cellcolor{verde}{P2P2.0}&2&9&2 \\ \hline
	\cellcolor{verde}{P2P2.1}&2&9&2 \\ \hline
	\cellcolor{verde}{P2P2.2}&2&9&2 \\ \hline
	\cellcolor{verde}{P2P2.3}&2&9&2 \\ \hline
	\cellcolor{verde}{P2P2.4}&2&9&2 \\ \hline
    \end{tabular}
    \caption{Tabla de subredes de la organización B.}
    \label{tab:orgB.1}
\end{table}
\par El siguiente paso es asignar los rangos de direcciones para cada \textbf{área}:
\begin{itemize}
	\item{\textbf{ÁREA 0}} = $256 + 128 + 4 + 4 + 4 = 396$. Se necesitan 9 bits para la máscara $\rightarrow$ \textbf{173.89.8.0/23}
	\item{\textbf{ÁREA 1}} = $128 + 32 = 160$. Se necesitan 8 bits para la máscara $\rightarrow$ \textbf{173.89.10.0/24}
	\item{\textbf{ÁREA 2}} = $128 + 64 + 4 + 4 = 200$. Se necesitan 8 bits para la máscara $\rightarrow$ \textbf{173.89.11.0/24}
\end{itemize}

%%%% TABLA REPARTO DE DIRECCIONES ORGANIZACIÓN A %%%%
\definecolor{marron}{rgb}{0.43, 0.21, 0.1}
\definecolor{verdeoscuro}{rgb}{0.0, 0.42, 0.24}
\definecolor{lilaclaro}{rgb}{0.96, 0.73, 1.0}

\begin{table}[H]
    \centering
    \begin{tabular}{|c|c|c|c|c|c|c|c|c|c|}
        \hline
       	Áreas&&+0 &+4&+8&+12&+16&+20&+24&+28\\\hline
	\cellcolor{marron}{ }&0&\multicolumn{8}{c}{\cellcolor{rojo}{ }}\\
	\cellcolor{marron}{ }&32&\multicolumn{8}{c}{\cellcolor{rojo}{ }}\\
	\cellcolor{marron}{ }&...&\multicolumn{8}{c}{\cellcolor{rojo}{LAN2.0}}\\
	\cellcolor{marron}{ }&224&\multicolumn{8}{c}{\cellcolor{rojo}{ }}\\\hline 
	\cellcolor{marron}{ }&256&\multicolumn{8}{c}{\cellcolor{naranja}{ }}\\ 
	\cellcolor{marron}{0}&...&\multicolumn{8}{c}{\cellcolor{naranja}{LAN2.1}}\\
	\cellcolor{marron}{ }&352&\multicolumn{8}{c}{\cellcolor{naranja}{ }}\\ \hline
	\cellcolor{marron}{ }&384&\cellcolor{verde}{P2P2.0} &\cellcolor{verde}{P2P2.1} &\cellcolor{verde}{P2P2.2}&&&&&\\ \hline
	\cellcolor{marron}{ }&...&&&&&&&&\\ \hline
	\cellcolor{marron}{ }&480&&&&&&&&\\ \hline \hline
	\cellcolor{verdeoscuro}{ }&512&\multicolumn{8}{c}{\cellcolor{lila}{ }} \\ 
	\cellcolor{verdeoscuro}{ }&...&\multicolumn{8}{c}{\cellcolor{lila}{ LAN2.2}}  \\ 
	\cellcolor{verdeoscuro}{ }&608&\multicolumn{8}{c}{\cellcolor{lila}{}}  \\ \hline
	\cellcolor{verdeoscuro}{ 1}&640&\multicolumn{8}{c}{\cellcolor{rosa}{LAN2.3}}  \\ \hline
	\cellcolor{verdeoscuro}{ }&...&&&&&&&&\\ \hline
	\cellcolor{verdeoscuro}{ }&736&&&&&&&&\\ \hline \hline
	\cellcolor{lilaclaro}{ }&768&\multicolumn{8}{c}{\cellcolor{amarillo}{ }} \\
	\cellcolor{lilaclaro}{ }&...&\multicolumn{8}{c}{\cellcolor{amarillo}{LAN2.4}} \\
	\cellcolor{lilaclaro}{}&864&\multicolumn{8}{c}{\cellcolor{amarillo}{ }} \\ \hline
	\cellcolor{lilaclaro}{2}&896&\multicolumn{8}{c}{\cellcolor{azul}{ }} \\
	\cellcolor{lilaclaro}{}&928&\multicolumn{8}{c}{\cellcolor{azul}{LAN2.5}} \\ \hline
	\cellcolor{lilaclaro}{ }&960&\cellcolor{verde}{P2P2.3} &\cellcolor{verde}{P2P2.4}&&&&&&\\ \hline
	\cellcolor{lilaclaro}{ }&992&&&&&&&&\\ \hline
    \end{tabular}
    \caption{Tabla reparto de direcciones organización B.}
    \label{tab:orgB.2}
\end{table}

\par \colorbox{marron}{\textbf{\textcolor{white}{ÁREA 0: Se pueden agregar bajo la dirección 173.89.8.0/23}}}
\begin{itemize}
	%% Subred LAN2.0 %%
	\item{\textbf{Subred LAN2.0}:} 173.89.\{\textcolor{azul}{000010}\textcolor{rojo}{00}\}.\{00000000\}, es decir, \textbf{173.89.8.0/24}
		\begin{itemize}
			\item{\textbf{Dirección Broadcast}}: 173.89.\{\textcolor{azul}{000010}\textcolor{rojo}{00}\}.\{11111111\} = 173.89.8.255
			\item{\textbf{Dirección IP RouterA0-eth0}}: 173.89.\{\textcolor{azul}{000010}\textcolor{rojo}{00}\}.\{0000001\} = 173.89.8.1
			\item{\textbf{Dirección IP HostA1}}: 173.89.\{\textcolor{azul}{000010}\textcolor{rojo}{00}\}.\{00000010\} = 173.89.8.2
		\end{itemize}
	%% Subred LAN2.1 %%
	\item{\textbf{Subred LAN2.1}:} 173.89.\{\textcolor{azul}{000010}\textcolor{rojo}{01}\}.\{\textcolor{rojo}{0}0000000\}, es decir, \textbf{173.89.9.0/25}
		\begin{itemize}
			\item{\textbf{Dirección Broadcast}}: 173.89.\{\textcolor{azul}{000010}\textcolor{rojo}{01}\}.\{\textcolor{rojo}{0}1111111\} = 173.89.9.127
			\item{\textbf{Dirección IP RouterA0-eth0}}:173.89.\{\textcolor{azul}{000010}\textcolor{rojo}{01}\}.\{\textcolor{rojo}{0}0000001\} = 173.89.9.1
			\item{\textbf{Dirección IP HostA1}}:173.89.\{\textcolor{azul}{000010}\textcolor{rojo}{01}\}.\{\textcolor{rojo}{0}0000010\} = 173.89.9.2
		\end{itemize}
	%% Subred P2P2.0 %%
	\item{\textbf{Subred P2P2.0}:} 173.89.\{\textcolor{azul}{000010}\textcolor{rojo}{01}\}.\{\textcolor{rojo}{100000}00\}, es decir, \textbf{173.89.9.128/30}
		\begin{itemize}
			\item{\textbf{Dirección Broadcast}}: 173.89.\{\textcolor{azul}{000010}\textcolor{rojo}{01}\}.\{\textcolor{rojo}{100000}11\} = 173.89.9.131
			\item{\textbf{Dirección IP RouterA0-serial0}}: 173.89.\{\textcolor{azul}{000010}\textcolor{rojo}{01}\}.\{\textcolor{rojo}{100000}01\} = 173.89.9.129
			\item{\textbf{Dirección IP RouterA1-serial0}}: 173.89.\{\textcolor{azul}{000010}\textcolor{rojo}{01}\}.\{\textcolor{rojo}{100000}10\} = 173.89.9.130
		\end{itemize}
	%% Subred P2P2.1 %%
	\item{\textbf{Subred P2P2.1}:}173.89.\{\textcolor{azul}{000010}\textcolor{rojo}{01}\}.\{\textcolor{rojo}{100001}00\}, es decir, \textbf{173.89.9.132/30}
		\begin{itemize}
			\item{\textbf{Dirección Broadcast}}: 173.89.\{\textcolor{azul}{000010}\textcolor{rojo}{01}\}.\{\textcolor{rojo}{100001}11\} = 173.89.9.135
			\item{\textbf{Dirección IP RouterA0-serial0}}: 173.89.\{\textcolor{azul}{000010}\textcolor{rojo}{01}\}.\{\textcolor{rojo}{100001}01\} = 173.89.9.133
			\item{\textbf{Dirección IP RouterA1-serial0}}: 173.89.\{\textcolor{azul}{000010}\textcolor{rojo}{01}\}.\{\textcolor{rojo}{100001}10\} = 173.89.9.134
		\end{itemize}
	%% Subred P2P2.2 %%
	\item{\textbf{Subred P2P2.2}:}173.89.\{\textcolor{azul}{000010}\textcolor{rojo}{01}\}.\{\textcolor{rojo}{110010}00\}, es decir, \textbf{173.89.9.136/30}
		\begin{itemize}
			\item{\textbf{Dirección Broadcast}}: 173.89.\{\textcolor{azul}{000010}\textcolor{rojo}{01}\}.\{\textcolor{rojo}{100010}11\} = 173.89.9.139
			\item{\textbf{Dirección IP RouterA0-serial0}}: 173.89.\{\textcolor{azul}{000010}\textcolor{rojo}{01}\}.\{\textcolor{rojo}{100010}01\} = 173.89.9.137
			\item{\textbf{Dirección IP RouterA1-serial0}}: 173.89.\{\textcolor{azul}{000010}\textcolor{rojo}{01}\}.\{\textcolor{rojo}{100010}10\} = 173.89.9.138
		\end{itemize}
\end{itemize}


\par \colorbox{verdeoscuro}{\textbf{\textcolor{white}{ÁREA 1: Se pueden agregar bajo la dirección 173.89.10.0/24}}}
\begin{itemize}
	%% Subred LAN2.2 %%
	\item{\textbf{Subred LAN2.2}:} 173.89.\{\textcolor{azul}{000010}\textcolor{rojo}{10}\}.\{\textcolor{rojo}{0}0000000\}, es decir, \textbf{173.89.10.0/25}
		\begin{itemize}
			\item{\textbf{Dirección Broadcast}}: 173.89.\{\textcolor{azul}{000010}\textcolor{rojo}{10}\}.\{\textcolor{rojo}{0}1111111\} = 173.89.10.127
			\item{\textbf{Dirección IP RouterA0-eth0}}: 173.89.\{\textcolor{azul}{000010}\textcolor{rojo}{10}\}.\{\textcolor{rojo}{0}000001\} = 173.89.10.1
			\item{\textbf{Dirección IP HostA1}}: 173.89.\{\textcolor{azul}{000010}\textcolor{rojo}{10}\}.\{\textcolor{rojo}{0}0000010\} = 173.89.10.2
		\end{itemize}
	%% Subred LAN2.3 %%
	\item{\textbf{Subred LAN2.3}:} 173.89.\{\textcolor{azul}{000010}\textcolor{rojo}{10}\}.\{\textcolor{rojo}{100}00000\}, es decir, \textbf{173.89.10.128/27}
		\begin{itemize}
			\item{\textbf{Dirección Broadcast}}: 173.89.\{\textcolor{azul}{000010}\textcolor{rojo}{10}\}.\{\textcolor{rojo}{100}11111\} = 173.89.10.255
			\item{\textbf{Dirección IP RouterA0-eth0}}:173.89.\{\textcolor{azul}{000010}\textcolor{rojo}{10} \}.\{\textcolor{rojo}{100}00001\} = 173.89.10.129
			\item{\textbf{Dirección IP HostA1}}:173.89.\{\textcolor{azul}{000010}\textcolor{rojo}{10} \}.\{\textcolor{rojo}{100}00010\} = 173.89.10.130
		\end{itemize}
\end{itemize}

\par \colorbox{lilaclaro}{\textbf{ÁREA 2: Se pueden agregar bajo la dirección 173.89.11.0/24}}
\begin{itemize}
	%% Subred LAN2.4 %%
	\item{\textbf{Subred LAN2.4}:} 173.89.\{\textcolor{azul}{000010}\textcolor{rojo}{11}\}.\{\textcolor{rojo}{0}0000000\}, es decir, \textbf{173.89.11.0/25}
		\begin{itemize}
			\item{\textbf{Dirección Broadcast}}: 173.89.\{\textcolor{azul}{000010}\textcolor{rojo}{11}\}.\{\textcolor{rojo}{0}1111111\} = 173.89.11.127
			\item{\textbf{Dirección IP RouterA0-eth0}}: 173.89.\{\textcolor{azul}{000010}\textcolor{rojo}{11}\}.\{\textcolor{rojo}{0}000001\} = 173.89.11.1
			\item{\textbf{Dirección IP HostA1}}: 173.89.\{\textcolor{azul}{000010}\textcolor{rojo}{11}\}.\{\textcolor{rojo}{0}0000010\} = 173.89.11.2
		\end{itemize}
	%% Subred LAN2.5 %%
	\item{\textbf{Subred LAN2.5}:} 173.89.\{\textcolor{azul}{000010}\textcolor{rojo}{11}\}.\{\textcolor{rojo}{10}000000\}, es decir, \textbf{173.89.11.128/26}
		\begin{itemize}
			\item{\textbf{Dirección Broadcast}}: 173.89.\{\textcolor{azul}{000010}\textcolor{rojo}{11}\}.\{\textcolor{rojo}{10}111111\} = 173.89.11.191
			\item{\textbf{Dirección IP RouterA0-eth0}}: 173.89.\{\textcolor{azul}{000010}\textcolor{rojo}{11}\}.\{\textcolor{rojo}{10}00001\} = 173.89.11.129
			\item{\textbf{Dirección IP HostA1}}: 173.89.\{\textcolor{azul}{000010}\textcolor{rojo}{11}\}.\{\textcolor{rojo}{10}000010\} = 173.89.11.130
		\end{itemize}
	%% Subred P2P2.3 %%
	\item{\textbf{Subred P2P2.3}:} 173.89.\{\textcolor{azul}{000010}\textcolor{rojo}{11}\}.\{\textcolor{rojo}{110000}00\}, es decir, \textbf{173.89.11.192/30}
		\begin{itemize}
			\item{\textbf{Dirección Broadcast}}: 173.89.\{\textcolor{azul}{000010}\textcolor{rojo}{11}\}.\{\textcolor{rojo}{110000}11\} = 173.89.11.195
			\item{\textbf{Dirección IP RouterA0-serial0}}: 173.89.\{\textcolor{azul}{000010}\textcolor{rojo}{11}\}.\{\textcolor{rojo}{110000}01\} = 173.89.11.193
			\item{\textbf{Dirección IP RouterA1-serial0}}: 173.89.\{\textcolor{azul}{000010}\textcolor{rojo}{11}\}.\{\textcolor{rojo}{110000}10\} = 173.89.11.194
		\end{itemize}
	%% Subred P2P2.4 %%
	\item{\textbf{Subred P2P2.4}:}173.89.\{\textcolor{azul}{000010}\textcolor{rojo}{11}\}.\{\textcolor{rojo}{110001}00\}, es decir, \textbf{173.89.11.196/30}
		\begin{itemize}
			\item{\textbf{Dirección Broadcast}}: 173.89.\{\textcolor{azul}{000010}\textcolor{rojo}{11}\}.\{\textcolor{rojo}{110001}11\} = 173.89.11.199
			\item{\textbf{Dirección IP RouterA0-serial0}}: 173.89.\{\textcolor{azul}{000010}\textcolor{rojo}{11}\}.\{\textcolor{rojo}{110001}01\} = 173.89.11.197
			\item{\textbf{Dirección IP RouterA1-serial0}}: 173.89.\{\textcolor{azul}{000010}\textcolor{rojo}{11}\}.\{\textcolor{rojo}{110001}10\} = 173.89.11.198
		\end{itemize}
\end{itemize}






